\section*{Activitiy 1}

\begin{enumerate}

    \item The main vulnerability on the \texttt{Low} setting is that it executes a ping command and appends whatever is in the textbox to the end of that command, without any input validation.\\
    This gives the attacker the capability of executing any commands as the same user as the application, effectively giving access to a backdoor terminal by using command concatenation.\\
    For example, by inserting \texttt{; whoami ; pwd} (resulting in the command \texttt{ping -c ; whoami ; pwd}\footnote{The \texttt{;} concatenator in a Linux shell allows the execution of a command regardless of the output of the previous command. In this case, the \texttt{ping} command will fail, but it will still execute the rest of the commands.}), we can obtain the user the shell is executing as (\texttt{www-data}) and the current working directory (\texttt{/var/www/dvwa/vulnerabilities/exec}).
    \item The \texttt{Medium} setting improves things by filtering out the \texttt{;} and \texttt{\&\&} concatenators, but this still leaves some things, like pipes or redirection, enabled.\\
    We can directly use a pipe to execute any command\footnote{As a pipe only redirects \texttt{stdout} and not \texttt{stderr}, by purposely making the \texttt{ping} command fail by not giving it an argument, enabling us to effectively concatenate another command.} e.g. \texttt{| whoami} (resulting in \texttt{ping -c | whoami}).\newline
    
    The \texttt{High} setting is even more strict, by validating the input. It splits the input into an array by using the \texttt{.} separator and checks that the first four numbers are integers, re-generating the IP using only those first four numbers.\\
    I didn't find any way to exploit it.
    
\end{enumerate}


\section*{Activitiy 2}

\begin{enumerate}
    \item 
\end{enumerate}