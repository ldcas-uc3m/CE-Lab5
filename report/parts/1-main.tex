\section*{Activity 1}

\begin{enumerate}

    \item The main vulnerability on the \texttt{Low} setting is that it executes a ping command and appends whatever is in the textbox to the end of that command, without any input validation.\\
    This gives the attacker the capability of executing any commands as the same user as the application, effectively giving access to a backdoor terminal by using command concatenation.\\
    For example, by inserting \texttt{; whoami ; pwd} (resulting in the command \texttt{ping -c ; whoami ; pwd}\footnote{The \texttt{;} concatenator in a Linux shell allows the execution of a command regardless of the output of the previous command. In this case, the \texttt{ping} command will fail, but it will still execute the rest of the commands.}), we can obtain the user the shell is executing as (\texttt{www-data}) and the current working directory (\texttt{/var/www/dvwa/vulnerabilities/exec}).
    \item The \texttt{Medium} setting improves things by filtering out the \texttt{;} and \texttt{\&\&} concatenators, but this still leaves some things, like pipes or redirection, enabled.\\
    We can directly use a pipe to execute any command\footnote{As a pipe only redirects \texttt{stdout} and not \texttt{stderr}, by purposely making the \texttt{ping} command fail by not giving it an argument, enabling us to effectively concatenate another command.} e.g. \texttt{| whoami} (resulting in \texttt{ping -c | whoami}).\newline
    
    The \texttt{High} setting is even more strict, by validating the input. It splits the input into an array by using the \texttt{.} separator and checks that the first four numbers are integers, re-generating the IP using only those first four numbers.\\
    I didn't find any way to exploit it, without modifying the packages so that it doesn't use that script.
    
\end{enumerate}


\section*{Activity 2}

\begin{enumerate}
    \item The PHP code outputs just the information of the columns \texttt{first\_name} and \texttt{last\_name}.\\
    By using unions, we can trick the server into renaming columns of the table \\\texttt{information\_schema.tables}, which contains the names of all tables, to those the PHP script will display.\\
    Inputting \texttt{' AND 1=0 UNION SELECT table\_name, NULL FROM information\_schema.tables \#} (resulting in the query \texttt{SELECT first\_name, last\_name FROM users WHERE user\_id = '' AND 1=0 UNION SELECT table\_name, NULL FROM information\_schema.tables \#'}), will make MySQL select no rows from \texttt{users} (\texttt{1=0} is always false), and put all the table names (and \texttt{null}, nothing, in the second column) in the correct columns. The \texttt{\#} is for commenting out the last \texttt{'}.
    \item Modifying the previous query by appending \texttt{WHERE table\_name LIKE 'user\%'} (resulting in the query \texttt{SELECT first\_name, last\_name FROM users WHERE user\_id = '' AND 1=0 UNION SELECT table\_name, NULL FROM information\_schema.tables WHERE table\_name LIKE 'user\%' \#'}) we can output just the tables that start with "user". This returns, among others, the table \texttt{user}.\\
    To output the columns of this table, we can use \texttt{' AND 1=0 UNION SELECT column\_name, NULL FROM information\_schema.columns WHERE table\_name = 'users' \#}, finding out the columns of that table are \texttt{user\_id}, \texttt{first\_name}, \texttt{last\_name}, \texttt{user}, \texttt{password}, and \texttt{avatar}.
    \item Following the previous command structure, we use \texttt{' AND 1=0 UNION SELECT user, password FROM users \#} in order to get the users and passwords\footnote{You can replace one of the column names to \texttt{concat(user, ":", password)} in order to get a "cleaner" format for JTR.}.\\
    This outputs the following users and (hashed) passwords:
    \begin{itemize}
        \item \texttt{admin:5f4dcc3b5aa765d61d8327deb882cf99}
        \item \texttt{gordonb:e99a18c428cb38d5f260853678922e03}
        \item \texttt{1337:8d3533d75ae2c3966d7e0d4fcc69216b}
        \item \texttt{pablo:0d107d09f5bbe40cade3de5c71e9e9b7}
        \item \texttt{smithy:5f4dcc3b5aa765d61d8327deb882cf99}
    \end{itemize}
    \item By using JohnTheRipper, we can crack those passwords, resulting in the following:
    \begin{itemize}
        \item \texttt{admin:password}
        \item \texttt{gordonb:abc123}
        \item \texttt{1337:charley}
        \item \texttt{pablo:letmein}
        \item \texttt{smithy:password}
    \end{itemize}
\end{enumerate}



\section*{Activity3}

\begin{enumerate}
    \item 
\end{enumerate}